\documentclass{beamer}
\usepackage{mathtools}
\usepackage[utf8]{inputenc}
\usetheme{Warsaw}
\usecolortheme{beaver}
\title{Conceitos básicos de realidade aumentada sem marcadores}
\author{Mailson D. Lira Menezes}
\institute{Universidade Federal de Pernambuco}
\date{23 de outubro de 2012}
\begin{document}

\begin{frame}
\titlepage
\end{frame}

\begin{frame}
    Realidade aumentada
    \begin{itemize}
        \item Computação gráfica + Processamento de imagens + Visão computacional
        \item Inserção de elementos virtuais em cenas reais
    \end{itemize}
\end{frame}

% TODO: frame com imagem de realidade aumentada tradicional (com marcador)

\begin{frame}{Câmera}
    \textbf{Parâmetros intrínsecos} são aqueles inerentes à câmera, independentes de posição\\
    % $
    % k = 
    % \begin{pmatrix}
    %     a_x & s   & u_0 \\
    %     0   & a_y & v_0 \\
    %     0   & 0   & 1
    % \end{pmatrix}
    % $
    \begin{itemize}
        \item Distância focal
        \item Quantidade de pixels por unidade de distância
    \end{itemize}
    \textbf{Parâmetros extrínsecos} são dados em função da localização no espaço\\
    \begin{itemize}
        \item Rotação
        \item Translação
    \end{itemize}
\end{frame}

\begin{frame}
    \textbf{Matriz de pose} é obtida a partir da composição dos parâmetros intrínsecos e extrínsecos da câmera
\end{frame}

\begin{frame}
    Realidade aumentada sem marcadores
    \begin{itemize}
        \item Utiliza elementos naturalmente presentes na cena
        \item SfM/SLAM
        \item Técnica baseada em modelos
    \end{itemize}
\end{frame}

\begin{frame}
    SfM/SLAM
    \begin{itemize}
        \item Recuperação da pose da câmera durante o rastreamentos
    \end{itemize}
\end{frame}

\begin{frame}
    Cálculo de pose bottom-up
    \begin{itemize}
        \item Perspective-n-points
        \item RanSaC
    \end{itemize}
\end{frame}


% -----------------------------------
% \begin{frame}{Introduction}
% This is a short introduction to Beamer class.
% \end{frame}
% 
% \section{Section 1}
% \begin{frame}
%     Hello
%     \begin{itemize}
%         \pause \item Testing beamer
%         \pause \item This is a good test
%     \end{itemize}
% \end{frame}
% 
% \begin{frame}
%     aye
%     \begin{itemize}
%         \item<2-> appears from slide 2 on
%         \item<2-4> appears from slide 3 on
%         \item<4-> appears from slide 4 on
%         \item<5-> appears from slide 5 on
%     \end{itemize}
% \end{frame}
% 
% \section{Section 2}
% \begin{frame}
%     oie
%     \begin{itemize}[<+->]
%         \item L
%         \item A
%         \item T
%     \end{itemize}
% \end{frame}
% 
% \begin{frame}
% 
% \begin{itemize}
% \item Language used by Beamer: L\uncover<2->{A}TEX
% \item Language used by Beamer: L\only<2->{A}TEX
% \end{itemize}
% 
% \begin{block}{Block title}
% This is a block in blue
% \end{block}
% 
% \begin{alertblock}{Alert-block title}
% This is a block in red
% \end{alertblock}
% 
% \begin{exampleblock}{Example-block title}
% This is a block in green
% \end{exampleblock}
% 
% \end{frame}

\end{document}
