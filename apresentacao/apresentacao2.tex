\documentclass{beamer}
\usepackage{mathtools}
\usepackage[utf8]{inputenc}
\usetheme{default}
\usecolortheme{beaver}
\title{MAR baseada em arestas}
\author{Mailson D. Lira Menezes}
\institute{Universidade Federal de Pernambuco}
\date{29 de outubro de 2012}
\begin{document}

\begin{frame}
\titlepage
\end{frame}

\begin{frame}
    \begin{itemize}
        \item Técnica baseada em modelo
        \item Eficiente
        \item Relativamente simples
        \item Admite mudança brusca de iluminação
        \item Não admite mudança brusca de câmera
    \end{itemize}
\end{frame}

\begin{frame}{Técnicas}
    \begin{itemize}
        \item Extração explícita de arestas
        \item Amostragem de pontos
    \end{itemize}
\end{frame}

\begin{frame}{Extração explícita de arestas}
    \begin{enumerate}
        \item Segmentação da imagem (extração da aresta)
        \item Correspondência modelo-cena
            \begin{itemize}
                \item Distância de Mahalonabis \\
                    Cada aresta do modelo é relacionada com uma aresta da imagem que possui a menos distância de Mahalonabis
                    \[
                        d = (X_i - X_v)^t(\textrm{Cov})^{-1}(X_i - X_v)
                    \]
            \end{itemize}
        \item Estimativa de pose \\
            Minimizar o erro utilizando o algoritmo de Levenberg-Marquardt
            \[
                \textrm{erro} = \sum_j(X^j_i - X^j_v(P))^t(\textrm{Cov})^{-1}(X^j_i - X^j_v(P))
            \]
    \end{enumerate}
\end{frame}

\begin{frame}{Amostragem de pontos}
    \begin{itemize}
        \item Relaciona amostras de pontos 3D com pontos 2D extraídos da cena
        \item Precisa de uma estimativa inicial de pose
        \item Técnicas
            \begin{itemize}
                \item Drummond
                \item Wuest
                \item Moving Edges
                \item Céline
            \end{itemize}
    \end{itemize}
\end{frame}

\begin{frame}{Técnica de Drummond}
    \begin{enumerate}
        \item Renderiza modelo com a pose anterior
        \item Extração dos pontos visíveis no modelos \\
            BSP
        \item Fazer correspondência dos pontos do modelo com pontos da imagem \\
            Busca ao longo da normal da aresta
        \item Cálculo da estimativa de movimento \\
            Utiliza um estimador robusto para eliminar os \emph{outliers}
        \item Cálculo da pose a partir da estimativa de movimento
    \end{enumerate}
\end{frame}

\begin{frame}{Técnica de Wuest}
    É uma variação mais simples e otimizada do trabalho de Drummond
    \begin{enumerate}
        \item Amostragem proporcional de pontos
            \[
                n = \textrm{tamanho da linha de projeção} \times \textrm{coeficiente de amostragem}
            \]
        \item Extração dos pontos visíveis \\
            Extrai arestas a partir dos pontos visíveis
        \item Encontra pontos de forte gradiente (hipóteses) como prováveis correspondentes 2D-3D
        \item Estimativa de pose
            \begin{itemize}
                \item Múltiplas hipóteses
                \item Modelo adaptativo
                \item Algoritmo de minimização Levenberg-Marquardt
            \end{itemize}
    \end{enumerate}
\end{frame}

\begin{frame}{Técnica com Moving Edges}
    Variação da técnica de Wuest que usa o algoritmo Moving Edges para encontrar os pontos correspondentes
    \begin{enumerate}
        \item Amostragem proporcional de pontos
        \item Extração dos pontos visíveis
        \item Encontrar hipóteses a partir do Moving Edges
    \end{enumerate}
\end{frame}

\begin{frame}{Moving Edges}
    Para cada aresta $E_i$ projetada com a pose anterior, traçamos uma normal passando pelo ponto amostrado $e_{i,j}$. As hipóteses de correspondência para $e_{i,j}$ são os pontos com gradiente maior que $g$ e que estão dentro de um intervalo de busca pré-definido. \\
    \medskip
    Caso seja utilizada a abordagem de uma única hipótese, a hipótese com maior gradiente é retornada.
\end{frame}

\begin{frame}{Técnica com Moving Edges}
    Variação da técnica de Wuest que usa o algoritmo Moving Edges para encontrar os pontos correspondentes
    \begin{enumerate}
        \item Amostragem proporcional de pontos
        \item Extração dos pontos visíveis
        \item Encontrar hipóteses a partir do Moving Edges
        \item Estimativa de pose
            \begin{itemize}
                \item Minimização do erro de reprojeção ($r$)
                    \begin{itemize}
                        \item Única hipótese: $r = \sum^n_{i=0}{\textrm{dist}(P \ast M_i, m_i)}$
                        \item Múltiplas hipóteses: $r = \sum^n_{i=0}{\textrm{dist}(P \ast M_i, \textrm{mindist}(m^j_i, P \ast M_i))}$
                    \end{itemize}
            \end{itemize}
    \end{enumerate}
\end{frame}

\begin{frame}{Técnica de Céline com múltiplas hipóteses}
    \begin{enumerate}
        \item Projetar modelo com a pose anterior
        \item Para cada amostra $e_{i,j}$ da aresta $E_i$ é extraído um conjunto de hipóteses $\{e'_{i,j,l}\}$
        \item As hipóteses de cada aresta são classificadas utilizando a clusterização \emph{k-means}
    \end{enumerate}
\end{frame}

\begin{frame}{K-means}
    \begin{itemize}
        \item Para cada aresta $E_i$, as hipóteses são separadas em $k_i$ classes $(c^i_1, \dots , c^i_{k_i})$. Cada classe é um conjunto de pontos e o seu centróide é a reta obtida pelos pontos da classe.
        \item O algoritmo é inicializado com $k_i = \textrm{max}_j\{n_{i,j}\}$, sendo $n_{i,j}$ o número de hipóteses selecionadas para a amostra $e_{i,j}$.
        \item A m-ésima classe é inicializada com a m-ésima hipótese encontrada em cada aresta, ou seja $c^i_m = \{e'_{i,j,m}\}_j$
        \item A cada iteração o centróide de cada classe é computado
        \item Cada hipótese é então reagrupada para o \emph{cluster} mais próximo
    \end{itemize}
\end{frame}

\begin{frame}{Dúvidas}
    \begin{itemize}
        \item O artigo diz que após o \emph{k-means} temos o conjunto de classes $c^i_m = (\{e'_{i,j,m}\}_j, r^i_m)$, onde $r^i_m$ é o resíduo do \emph{least square minimization}.
        \item Também diz que escolhas randômicas com pesos $w^i_m$ são feitas, sendo
            \[
            w^i_m = \begin{cases}
                e^{-\lambda \left( \frac{r^i_m - r^i_{min}}{r^i_{max} - r^i_{min}}\right)^2 } & \mbox{se } r^i_{max} \neq r^i_{min} \\
                1 & \mbox{senão}.
            \end{cases}
            \]
            onde $\lambda$ é um valor arbitrário
        \item O cálculo da pose é feito minimizando
            \[
                S = \frac{1}{N_e} \sum_i \sum_{e'_{i,j,l} \in c^i_{p_i}} \rho (\Delta_{E_i}(e'_{i,j,l}))
            \]
            onde $\Delta_{E_i}(e'_{i,j,l})$ é a distância ao quadrado entre a aresta $E_i$ e o ponto $e'_{i,j,l}$, $N_e$ é o número total de amostras e $\rho$ é um estimador robusto.
    \end{itemize}
\end{frame}

\end{document}
