\documentclass[12pt]{article}

\usepackage{sbc-template}
\usepackage{graphicx}
\usepackage{url}
\usepackage[brazil]{babel}
\usepackage[utf8]{inputenc}
\usepackage[hidelinks]{hyperref}

\hypersetup{
    pdfauthor={Mailson Daniel Lira Menezes},
    pdftitle={Aprimoramento da etapa de casamento de uma técnica de rastreamento baseado em arestas}
}

\sloppy

\title{Aprimoramento da etapa de casamento de uma técnica de rastreamento baseado em arestas}
\author{Mailson Daniel Lira Menezes\inst{1}}
\address{Centro de Informática -- Universidade Federal de Pernambuco
  (UFPE)\\
  Recife -- PE -- Brasil
  \email{mdlm@cin.ufpe.br}
}

\begin{document}

\maketitle

\section{Estrutura do artigo}

O artigo estará organizado da seguinte forma:

\begin{description}
    \item[Introdução] \hfill \\
    \begin{itemize}
        \item Visão geral do trabalho (justificativa, relevância, motivação)
        \item Objetivo do trabalho \\
    \end{itemize}

    \item[Conceitos básicos] \hfill \\
    \begin{itemize}
        \item Conceituar rastreamento \cite{lepetit}
        \item Desafios
        \item Rastreamento por aresta
        \item Funcionamento do rastreamento por aresta (de uma forma geral)
        \item Aplicação em realidade aumentada \\
    \end{itemize}

    \item[Descrição da técnica] \hfill \\
    Descrição da técnica utilizada no trabalho \cite{celine}.
\end{description}

\nocite{*}
\bibliographystyle{sbc}
\bibliography{../bibliography}

\end{document}
