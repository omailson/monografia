\chapter{Introdução}

O rastreamento 3D é um passo importante das técnicas de realidade aumentada. A partir de uma sequência de imagens 2D deseja-se saber em que posição do espaço 3D está o objeto rastreado, o que é uma tarefa importante e desafiadora. Várias técnicas de rastreamento 3D podem ser encontradas na literatura \cite{lepetit}, porém ainda existe muito a ser pesquisado na área. As técnicas existentes, apesar de apresentar uma certa estabilidade no rastreamento, ainda apresentam dificuldades em lidar com elementos externos no ambiente ou movimentação imprevisível da câmera utilizada no rastreamento.

Como será mostrado no capítulo seguinte, o objetivo das técnicas de rastreamento 3D é encontrar a configuração da câmera virtual que representa, da melhor forma, o posicionamento relativo entre a câmera utilizada para capturar os \emph{frames} e o objeto a ser rastreado. Neste trabalho de graduação será apresentada uma técnica que ao invés de calcular apenas uma pose a partir dos dados recuperados da imagem, calcula várias poses para um único \emph{frame}, e ao final somente a melhor delas é selecionada como a pose calculada para aquele \emph{frame}.

Este trabalho está dividido em cinco capítulos. No capítulo 2 são apresentados os conceitos básicos e técnicas já difundidas na literatura sobre rastreamento 3D. No capítulo 3 é feita uma descrição da abordagem utilizada neste trabalho, bem como alguns detalhes de implementação são explicados. No capítulo 4 são feitas análises de qualidade e desempenho da abordagem, e o capítulo 5 finaliza o trabalho com uma discussão final dos resultados dos experimentos.
