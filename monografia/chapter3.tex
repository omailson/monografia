\chapter{Descrição da técnica}

A técnica usada neste trabalho é baseada em arestas e usa um modelo pré-definido para a pose inicial. A extração das arestas é feita por amostragem de pontos, descrita na seção anterior.

\begin{comment}
\section{Trabalhando com múltiplas hipóteses}

Um dos problemas que ocorrem nas técnicas \emph{edge-based} é que nem sempre se consegue extrair com precisão as arestas da imagem. Na seção anterior foi discutido que um dos passos para a extração das arestas é a procura por pontos de alto gradiente na normal da aresta. Muitas vezes é possível encontrar vários desses pontos para cada amostra, e neste caso obtém-se múltiplas hipóteses de pontos da imagem correspondente ao ponto amostrado. Veja a \figref{cubo_0}.

%O que se faz neste caso é, para cada aresta do modelo, trabalhar com múltiplos correspondentes na imagem.

%Como existem múltiplas hipóteses de correspondência em cada amostra do modelo, pode-se concluir que existirão múltiplas hipóteses de arestas para compor a pose da cena atual, como mostra a \figref{cubo_0}

O trabalho com múltiplas hipóteses foi inicialmente proposto em \cite{multiplas_hipoteses}. Neste trabalho cada aresta $E_i$ (do modelo) projetada na imagem tem um conjunto $\{e_{i,j}\}$ de pontos amostrados. Cada ponto $e_{i,j}$ tem um conjunto $\{e'_{i,j,l}\}$ de hipóteses de correspondência. Em \cite{multiplas_hipoteses} a hipótese $e'_{i,j,l}$ é aquela que tem a menor distância da aresta projetada $E_i$. Então mesmo que sejam encontrados pontos de forte gradiente na normal da aresta $E_i$, aquele que mais se aproxima da aresta projetada é escolhido como correspondente a $e_{i,j}$. Tendo este conjunto de correspondências, o processo continua como no caso de hipótese única.
\end{comment}

Como foi visto anteriormente, a técnica de múltiplas hipóteses escolhe um dos candidatos para ser associado com a amostra do modelo sendo escolhido como correspondente o candidato que está mais próximo da amostra projetada. Entretanto isso nem sempre resulta na melhor pose final. Outros pontos de forte gradiente como objetos externos ou até uma outra aresta do próprio objeto podem confundir todo o cálculo da pose só porque estão mais próximos de uma determinada aresta do modelo, como mostra a \figref{multiplas_hipoteses_errada}. Uma forma de contornar essa situação seria também escolher outras hipóteses (não somente a mais próxima) e calcular novas poses. O trabalho proposto em \cite{celine} é exatamente calcular diversas poses para então escolher a melhor entre elas.

\begin{figure}[!ht]
\centering\includegraphics{monografia/multiplas_hipoteses_errada}
\caption{Casamento de hipóteses errado. O algoritmo encontrou múltiplas hipóteses para as amostras, mas optou por casar com as hipóteses erradas porque estavam mais próximas.}
\label{multiplas_hipoteses_errada}
\end{figure}

Combinar todos os possíveis casamentos amostra-hipótese não é uma estratégia válida, não só por questão de performance. Um outro problema dessa estratégia é que nem sempre um agrupamento de hipóteses revela uma possível aresta do objeto. Na \figref{agrupamento_errado_de_hipoteses} a escolha das hipóteses vermelhas claramente não resultaria em uma pose válida. Para resolver essa questão, as hipóteses poderiam ser escolhidas de forma que fossem o mais próximo possível de uma reta. Sendo assim, ao invés de combinar diferentes hipóteses de pontos amostrados uma boa estratégia é combinar diferentes hipóteses de aresta.

\begin{figure}[!ht]
\centering\includegraphics{monografia/agrupamento_errado_de_hipoteses}
\caption{Hipóteses agrupadas sem formar uma linha. Os pontos vermelhos são as amostras do modelo; os quadrados são as hipóteses de cada amostras; os quadrados de cor verde são as hipóteses escolhidas.}
\label{agrupamento_errado_de_hipoteses}
\end{figure}

\section{Extraindo a hipótese de aresta}

As hipóteses de aresta são formadas agrupando $k_i$ conjuntos de $e'_{i,j,l}$ (uma das hipóteses da amostra $e_{i,j}$, como mostra a \figref{multiplas_hipoteses_celine}) e extraindo a reta que mais se aproxima desse conjunto de pontos. O número $k_i$ de conjuntos é dado pelo maior número de hipóteses detectadas por uma amostra da aresta $E_i$, ou seja $k_i = max_j\{n_{i,j}\}$, sendo $n_{i,j}$ o número de candidatos associados à amostra $e_{i,j}$.

\begin{figure}[!ht]
\centering\includegraphics[width=0.9\textwidth]{monografia/multiplas_hipoteses_celine}
\caption{Visualização das múltiplas hipóteses. Figura retirada de \cite{celine}.}
\label{multiplas_hipoteses_celine}
\end{figure}

Para formar esses conjuntos, as hipóteses $e'_{i,j,l}$ de cada aresta são agrupadas utilizando um algoritmo de classificação \emph{k-means}. Para cada aresta $E_i$, o algoritmo agrupa as hipóteses de pontos $e'_{i,j,l}$ em $k_i$ conjuntos (ou classes), que será chamada de $(c^i_1, \dots, c^i_{k_i})$. O centróide de cada classe é a reta resultante do algoritmo \emph{fitline} \cite{fitline_doc} com o conjunto de pontos da classe.

Inicialmente as hipóteses $e'_{i,j,l}$ da aresta $E_i$ são agrupadas nas classes $(c^i_1, \dots, c^i_{k_i})$ na ordem em que elas foram encontradas na busca pela normal. Dessa forma a classe $c^i_m$ é formada inicialmente pelo conjunto $\{e'_{i,j,m}\}$, em que $0 < j \leq n_i$ e $n_i$ é o número de amostras da aresta $E_i$. Esse é um bom agrupamento inicial, já que boa parte das amostras já estão localizadas no seu \emph{cluster} final. Em seguida são calculadas as distâncias de cada hipótese $e'_{i,j,l}$ da amostra $e_{i,j}$ para cada um dos centróides $(c^i_1, \dots, c^i_{k_i})$. Esse cálculo de distâncias serve para realocar as hipóteses para os \emph{clusters} mais próximos.

No trabalho original \cite{celine} é dito que esse último passo seja repetido até que não haja mais trocas entre os \emph{clusters}, mas na prática precisou-se estabelecer um número máximo de iterações, pois muitas vezes o algoritmo repetia indefinidamente.

% TODO: Esses passos são melhor explicados no algoritmo abaixo.

Ao final das iterações a aresta $E_i$ possui um conjunto de \emph{clusters} $c^i_m = (\{e'_{i,j,m}\}, r^i_m)$, sendo que $r^i_m$ é descrito na equação.

\begin{equation}
r^i_m = \frac{1}{N} \sum^{N}_{j = 0} \Delta (e'_{i,j,m}, c^i_m)
\end{equation}

em que $N$ é o número de elementos do \emph{cluster} $c^i_m$, e $\Delta (e', c)$ é a função que calcula a distância do ponto $e'$ à reta $c$. O resíduo $r^i_m$ será usado para a próxima seção.

\section{Obtendo as hipóteses de pose}

Após extrair as hipóteses de aresta, as hipóteses de pose são obtidas ao se escolher, para cada aresta $E_i$, uma classe $c^i_{p_i}$. A escolha é feita de forma aleatória, mas considerando o peso $w^i_m$ de cada hipótese de aresta.

O peso $w^i_m$ é deduzido pela equação:

\begin{equation}
w^i_m = \begin{cases}
    e^{-\lambda \left( \frac{r^i_m - r^i_{min}}{r^i_{max} - r^i_{min}}\right)^2 } & \mbox{se } r^i_{max} \neq r^i_{min} \\
    1 & \mbox{senão}.
\end{cases}
\end{equation}

em que $\lambda$ é um parâmetro que pode ser ajustado.

Faz-se uma escolha aleatória com pesos para que todas as hipóteses de aresta tenham chance de serem escolhidas para formar uma pose, embora as arestas de maior peso tenham mais chances de serem escolhidas.

Após isso é calculado o erro de reprojeção de cada pose formada. A que possuir menor erro é considerada a pose da cena atual e será usada para as próximas iterações do algoritmo.

\begin{comment}
\section{Rascunho}

\begin{figure}[ht!]
\centering
\includegraphics{monografia/cubo_arestas}
\caption{}
\label{cubo_arestas}
\end{figure}

A \figref{cubo_arestas} ilustra que as hipóteses de aresta são extraídas a partir das hipóteses das amostras. A escolha das arestas foi feita de maneira que para cada amostra escolhida, a primeira hipótese encontrada irá compor a primeira aresta; a segunda hipótese fará parte da segunda aresta e assim por diante.

\begin{figure}[ht!]
\centering
\includegraphics{monografia/cubo_kmeans}
\caption{}
\label{cubo_kmeans}
\end{figure}

No algoritmo usado nesse trabalho, baseado em \cite{celine}, é feita uma clusterização usando \emph{k-means} para que as hipóteses fiquem o mais próximo possível das arestas formadas. Como ilustrado na \figref{cubo_kmeans}, as hipóteses da \figref{cubo_arestas} são realocadas para que cada conjunto hipóteses forme um \emph{cluster} (ou aresta).

\section{Escolha da pose}

Para cada aresta da pose anterior, uma das hipóteses de aresta é escolhida aleatoriamente e assim será formada a pose atual.
\end{comment}

\begin{comment}
\section{A FAZER}

\begin{enumerate}
\item Descrever o moving-edges. Mostrar que com múltiplas hipóteses a $n$-ésima hipótese de ponto vai corresponder à $n$-ésima hipótese de aresta.
\item Falar sobre \cite{celine}. As hipóteses de pontos vão formar arestas tal que elas fiquem as mais paralelas possíveis da aresta da cena atual.
\item colocar figuras para ilustrar
\end{enumerate}
\end{comment}
