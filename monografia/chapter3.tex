\section{Descrição da técnica}

A técnica usada neste trabalho é baseada em arestas e usa um modelo pré-definido para a pose inicial. A extração das arestas é feita por amostragem de pontos, descrita na seção anterior.

\subsection{Trabalhando com múltiplas hipóteses}

Um dos problemas que ocorrem nas técnicas \emph{edge-based} é que nem sempre se consegue extrair com precisão as arestas da imagem. Na seção anterior descrevemos que um dos passos para a extração das arestas é a procura por pontos de alto gradiente na normal da aresta. Muitas vezes é possível encontrar vários desses pontos para cada amostra e neste caso temos múltiplas hipóteses de pontos da imagem correspondente ao ponto amostrado. Veja a \figref{cubo_0}.

%O que se faz neste caso é, para cada aresta do modelo, trabalhar com múltiplos correspondentes na imagem.

%Como existem múltiplas hipóteses de correspondência em cada amostra do modelo, pode-se concluir que existirão múltiplas hipóteses de arestas para compor a pose da cena atual, como mostra a \figref{cubo_0}

\begin{figure}[ht!]
\centering
\includegraphics{monografia/cubo_exemplo}
\caption{}
\label{cubo_0}
\end{figure}

Na \figref{cubo_0}, o cubo azul representa o modelo da pose anterior; a imagem do cubo é a cena atual da qual pretende-se extrair a pose; os pontos vermelhos são as amostras do modelo; e os pontos em formato de X são as correspondências encontradas na cena atual. Note que devido a ruídos na imagem algumas amostras possuem mais de uma correspondente.

O trabalho com múltiplas hipóteses foi inicialmente proposto em \cite{multiplas_hipoteses}. Neste trabalho cada aresta $E_i$ (do modelo) reprojetada na imagem tem um conjunto $\{e_{i,j}\}$ de pontos amostrados. Cada ponto $e_{i,j}$ tem um conjunto $\{e'_{i,j,l}\}$ de hipóteses de correspondência. Em \cite{multiplas_hipoteses} a hipótese $e'_{i,j,l}$ é aquela que tem a menor distância da aresta reprojetada $E_i$. Então mesmo que sejam encontrados pontos de forte gradiente na normal da aresta $E_i$, aquele que mais se aproxima da aresta reprojetada é escolhido como correspondente a $e_{i,j}$. Tendo este conjunto de correspondências, o processo continua como no caso de hipótese única.

\subsection{Descrição do algoritmo}

O algoritmo proposto em \cite{celine} funciona de forma diferente. Ao invés de se trabalhar com múltiplas hipóteses dos pontos amostrados para se chegar a uma pose, trabalhamos com múltiplas hipóteses da própria pose.

\subsubsection{Extraindo a hipótese de aresta}

Tentar todas as combinações de hipóteses $e'_{i,j,l}$ possíveis para obter as hipóteses de pose não é uma estratégia válida. Para resolver esse problema agrupamos $k_i$ conjuntos de $e'_{i,j,l}$ (para cada aresta $E_i$), e de cada conjunto obtemos uma hipótese $E'_{i,k}$ de aresta.

Para formar esses conjuntos, as hipóteses $e'_{i,j,l}$ de cada aresta são agrupadas utilizando um algoritmo de classificação \emph{k-means}. Para cada aresta $E_i$, o algoritmo agrupa as hipóteses de pontos $e'_{i,j,l}$ em $k_i$ conjuntos (ou classes), que chamaremos de $(c^i_1, \dots, c^i_{k_i})$. O centróide de cada classe é a reta resultante do algoritmo \emph{fitline} \cite{fitline_doc} com o conjunto de pontos da classe.

Inicialmente agrupamos as hipóteses $e'_{i,j,l}$ da aresta $E_i$ nas classes $(c^i_1, \dots, c^i_{k_i})$ na ordem em que elas foram encontradas na busca pela normal. Dessa forma a classe $c^i_m$ é formada inicialmente pelo conjunto $\{e'_{i,j,m}\}$, em que $0 < j \leq n_i$ e $n_i$ é o número de amostras da aresta $E_i$. Esse é um bom agrupamento inicial, já que boa parte das amostras já estão localizadas no seu \emph{cluster} final. Em seguida são calculadas as distâncias de cada hipótese $e'_{i,j,l}$ da amostra $e_{i,j}$ para cada um dos centróides $(c^i_1, \dots, c^i_{k_i})$. Esse cálculo de distâncias serve para realocar as hipóteses para os \emph{clusters} mais próximos.

No trabalho original \cite{celine} é dito que esse último passo seja repetido até que não haja mais trocas entre os \emph{clusters}, mas na prática precisou-se estabelecer um número máximo de iterações, pois muitas vezes o algoritmo repetia indefinidamente.

% TODO: Esses passos são melhor explicados no algoritmo abaixo.

Ao final das iterações a aresta $E_i$ possui um conjunto de \emph{clusters} $c^i_m = (\{e'_{i,j,m}\}, r^i_m)$, em que $r^i_m$ é descrito na equação.

\begin{equation}
r^i_m = \frac{1}{N} \sum^{N}_{i = 0} \Delta (e'_{i,j,m}, c^i_m)
\end{equation}

em que $N$ é o número de amostras da aresta $E_i$, e $\Delta (e', c)$ é a função que calcula a distância do ponto $e'$ à reta $c$. O resíduo $r^i_m$ será usado para a próxima seção.

\subsection{Obtendo as hipóteses de pose}

Após obter as hipóteses de aresta, as hipóteses de pose são obtidas ao se escolher, para cada aresta $E_i$, uma classe $c^i_{p_i}$. A escolha é feita de forma aleatória, mas considerando o peso $w^i_m$ de cada hipótese de aresta.

O peso $w^i_m$ é deduzido pela equação:

\begin{equation}
w^i_m = \begin{cases}
    e^{-\lambda \left( \frac{r^i_m - r^i_{min}}{r^i_{max} - r^i_{min}}\right)^2 } & \mbox{se } r^i_{max} \neq r^i_{min} \\
    1 & \mbox{senão}.
\end{cases}
\end{equation}

em que $\lambda$ é um parâmetro que pode ser ajustado.

Faz-se uma escolha aleatória com pesos para que todas as hipóteses de aresta tenham chance de serem escolhidas para formar uma pose, embora as arestas de maior peso tenham mais chances de serem escolhidas.

Após isso é calculado o erro de reprojeção de cada pose formada. A que possuir menor erro é considerada a pose da cena atual e será usada para as próximas iterações do algoritmo.

\subsection{Rascunho}

\begin{figure}[ht!]
\centering
\includegraphics{monografia/cubo_arestas}
\caption{}
\label{cubo_arestas}
\end{figure}

Na \figref{cubo_arestas} extraímos as hipóteses de aresta a partir das hipóteses das amostras. A escolha das arestas foi feita de maneira que para cada amostra escolhida, a primeira hipótese encontrada irá compor a primeira aresta; a segunda hipótese fará parte da segunda aresta e assim por diante.

\begin{figure}[ht!]
\centering
\includegraphics{monografia/cubo_kmeans}
\caption{}
\label{cubo_kmeans}
\end{figure}

No algoritmo usado nesse trabalho, baseado em \cite{celine}, é feita uma clusterização usando \emph{k-means} para que as hipóteses fiquem o mais próximo possível das arestas formadas. Como ilustrado na \figref{cubo_kmeans}, as hipóteses da \figref{cubo_arestas} são realocadas para que cada conjunto hipóteses forme um \emph{cluster} (ou aresta).

\subsection{Escolha da pose}

Para cada aresta da pose anterior, uma das hipóteses de aresta é escolhida aleatoriamente e assim será formada a pose atual.

\begin{comment}
\subsection{A FAZER}

\begin{enumerate}
\item Descrever o moving-edges. Mostrar que com múltiplas hipóteses a $n$-ésima hipótese de ponto vai corresponder à $n$-ésima hipótese de aresta.
\item Falar sobre \cite{celine}. As hipóteses de pontos vão formar arestas tal que elas fiquem as mais paralelas possíveis da aresta da cena atual.
\item colocar figuras para ilustrar
\end{enumerate}
\end{comment}
