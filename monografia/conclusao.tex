\chapter{Conclusão}

A técnica de rastreamento 3D utilizando múltiplas hipóteses de pose, apesar de apresentar facilidade ao encontrar a melhor pose quando o modelo projetado se distancia muito do objeto rastreado, mostrou-se bastante instável na maioria dos experimentos, tanto no erro de reprojeção médio quanto na qualidade visual do rastreamento. A proposta de se calcular múltiplas poses para então escolher a melhor é interessante, entretanto a forma como essas hipóteses de pose são obtidas acaba por atrapalhar no resultado final, pois na maioria das vezes nenhuma das poses calculadas é boa se comparada à abordagem tradicional de rastreamento baseado em aresta.

Resultados melhores talvez possam ser encontrados se formas diferentes de se obter as hipóteses de pose forem usadas em conjunto com a baseada em clusterização \emph{k-means}. O trabalho apresentado por Teuliere \emph{et al.} \cite{celine} mostra que esta técnica em conjunto com um filtro de partículas apresenta resultados melhores que as técnicas baseadas em aresta tradicionais. Outra abordagem que talvez resulte em resultados melhores é utilizar como uma das hipóteses a pose calculada pela técnica baseada em arestas tradicional, em que as hipóteses de pontos associados à amostra são escolhidos seguindo um critério de distância ao ponto amostrado. Estas direções serão investigadas como trabalhos futuros desta pesquisa.
