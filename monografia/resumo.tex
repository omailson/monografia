\resumo
Este trabalho de graduação apresenta uma discussão sobre uma melhoria na etapa de casamento de múltiplas hipóteses de uma técnica de rastreamento 3D baseado em arestas. Essa melhoria visa facilitar a escolha das poses da câmera ao longo da sequência de vídeo para que se possa trabalhar com múltiplas hipóteses de pose no rastreamento. Neste trabalho são discutidas técnicas de rastreamento 3D e conceitos básicos necessários para o entendimentos do texto. Também é feita uma análise da técnica proposta em \cite{celine} e uma discussão dos resultados após a implementação da técnica. Após os experimentos conclui-se que a técnica tem uma certa instabilidade em manter uma pose correta, porém ela apresenta facilidade em encontrar uma pose mesmo quando o modelo projetado está bem distante do objeto rastreado.
\begin{keywords}
arestas, rastreamento, realidade aumentada
\end{keywords}

% \abstract
% \textbf{TODO:} abstract
% \begin{keywords}
% keywords, here
% \end{keywords}
