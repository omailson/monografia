\documentclass[a4paper, 12pt]{article}
\usepackage{graphicx,url}
\usepackage{color}
\usepackage{colortbl}
\usepackage[brazil]{babel}
\usepackage[utf8]{inputenc}
\usepackage[hidelinks]{hyperref}
\title{Aprimoramento da etapa de casamento de uma técnica de rastreamento baseado em arestas -- Proposta de trabalho de graduação}
\date{21 de janeiro de 2013}
\author{Mailson D. Lira Menezes}
\begin{document}
\maketitle

\section{Contexto}

Nas áreas de interação humano-computador e visão computacional a realidade aumentada (RA) tem sido bastante estudada nos últimos anos. Através dela pode-se inserir elementos virtuais em um ambiente real, promovendo uma maior imersão do usuário \cite{azuma1997survey}.

Das três áreas -- visualização, interação e rastreamento -- em que consiste a realidade aumentada, a que será discutida nesse trabalho é a última. Rastrear significa entender a cena (extraindo dados dela utilizando várias técnicas de visão computacional) a fim de que seja possível posicionar corretamente a informação virtual em tempo real.

A partir de uma captura de vídeo pode-se inserir um modelo 3D associado à cena sendo observada que acompanha o movimento da câmera. A técnica consiste em buscar uma aproximação do posicionamento da câmera real a fim de que o modelo 3D seja projetado corretamente. Geralmente são usados marcadores na cena para realizar este rastreamento \cite{roberto2012mestrado}, mas existem outras técnicas em que são utilizados objetos existentes na própria cena.

Por outro lado, é possível obter bons resultados utilizando realidade aumentada sem marcadores (em inglês Markerless Augmented Reality, ou MAR) utilizando elementos já presentes na cena \cite{teichrieb2007survey}. Um dos elementos que podem ser utilizados são as arestas do próprio objeto a ser rastreado. Com elas, pode-se identificar que houve um movimento na câmera e de que forma ele foi feito. A vantagem de MAR é que não é preciso utilizar elementos intrusivos na cena, pois o próprio objeto rastreado pode servir como guia.

Um dos passos do rastreamento baseado em aresta é extrair as bordas da imagem e, a partir delas, processar as arestas. Após a extração dar bordas é preciso casá-las com as arestas do modelo real. Algo que se pode admitir é que a aresta do modelo correspondente à borda extraída provavelmente é aquela que está mais perto dela e com uma inclinação semelhante, já que a diferença entre o frame atual e o anterior é, geralmente, pequena. O que se faz então é achar uma aresta correspondente no modelo e reposicionar a câmera a fim de que todas as arestas do modelo façam (no caso ideal) um match perfeito com as arestas da imagem no frame \cite{tgchico}.

No entanto, nem sempre uma única borda extraída casa com uma determinada aresta do modelo. Podem existir várias hipóteses de arestas da imagem capturada e isso é agravado caso existam ruídos na imagem como elementos adicionais na cena, ruídos da câmera ou até a sombra do próprio objeto.

Para melhorar este resultado será utilizada neste trabalho uma técnica que busca melhorar o casamento entre as arestas extraídas da imagem com as arestas do modelo \cite{celine}. Essa técnica, infelizmente, é relativamente lenta não sendo possível obter resultados em tempo real. Desta forma foi escolhido implementar a técnica em GPU, utilizando a linguagem CUDA, a fim de aproveitar a maior capacidade em realizar operações em paralelo desta tecnologia \cite{cuda} e, com isso, obter resultados mais adequados para aplicação em realidade aumentada.

\section{Objetivos}

O objetivo deste trabalho consiste em implementar em GPU a técnica proposta em \cite{celine} a fim de melhorar a robustez do rastreamento das arestas. O projeto será desenvolvido utilizando a linguagem CUDA a fim de aproveitar o poder do paralelismo das placas gráficas atuais e alcançar um desempenho de tempo real na execução da técnica.

Após a implementação, será realizada uma análise dos resultados verificando sua adequação para aplicação em realidade aumentada.

\section{Cronograma}

A tabela abaixo apresenta as atividades a serem realizadas durante o trabalho de graduação, bem como os prazos para finalização das mesmas.

\begin{table}[ht]
    \centering
    \begin{tabular}{| p{28ex} | l | l | l | l | l |}
        \hline
        Atividade                                            & Dezembro                 & Janeiro                  & Fevereiro                & Março                    & Abril                    \\
        \hline
        Levantamento e estudo do material bibliográfico      & \cellcolor[rgb]{1,0.6,0} & ~                        & ~                        & ~                        & ~                        \\
        \hline
        Levantamento e estudo dos desafios a serem abordados & \cellcolor[rgb]{1,0.6,0} & \cellcolor[rgb]{1,0.6,0} & ~                        & ~                        & ~                        \\
        \hline
        Implementação da técnica de rastreamento             & ~                        & \cellcolor[rgb]{1,0.6,0} & \cellcolor[rgb]{1,0.6,0} & \cellcolor[rgb]{1,0.6,0} & ~                        \\
        \hline
        Análise dos resultados                               & ~                        & ~                        & ~                        & \cellcolor[rgb]{1,0.6,0} & ~                        \\
        \hline
        Escrita da monografia                                & ~                        & ~                        & \cellcolor[rgb]{1,0.6,0} & \cellcolor[rgb]{1,0.6,0} & \cellcolor[rgb]{1,0.6,0} \\
        \hline
        Elaboração da apresentação oral                      & ~                        & ~                        & ~                        & ~                        & \cellcolor[rgb]{1,0.6,0} \\
        \hline
        Defesa do TG                                         & ~                        & ~                        & ~                        & ~                        & \cellcolor[rgb]{1,0.6,0} \\
        \hline
    \end{tabular}
\end{table}

\bibliographystyle{plain}
\bibliography{proposta}

\end{document}
